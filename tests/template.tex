\documentclass[aspectratio=169]{beamer}
% should also look nice for the classic aspectratio
% of course, than the text has to be refitted
% \documentclass{beamer} 
\usepackage[utf8]{inputenc}
\usepackage[T1]{fontenc}
\usetheme[]{pureminimalistic}
% \usepackage[backend=biber, doi=false, maxbibnames=2, maxcitenames=2,%
%             style=numeric, sorting=none, url=false, eprint=false]{biblatex}
% \addbibresource{demo_bib.bib}
% this makes it possible to add backup slides, without counting them
\usepackage{appendixnumberbeamer}
\renewcommand{\appendixname}{\texorpdfstring{\translate{appendix}}{appendix}}

% if loaded after begin{document} a warning will appear: "pdfauthor already used"
\title[short title]{Test template}
\author{Kai Norman Clasen}
\institute{Testing} 
\date{\today}

\begin{document}
% has to be loaded outside of a frame to work!
\maketitle

\begin{frame}[fragile]{Testing fragile frame}
  \begin{verbatim}
    Use the provided \vfilleditems environment to create 
    nicely spaced bullet points.

    \begin{vfilleditems}
      \item In my opinion, I like it to have my bullet points
      \item evenly spaced from one another
      \item then few bullet points, are not crammed on 
      the upper part of the slide
    \end{vfilleditems}
    \end{verbatim}
\end{frame}

\begin{frame}{Normal frame}
  This will show how a normal frame looks like
\end{frame}



\end{document}
